\documentclass[withoutpreface,bwprint]{cumcmthesis} %去掉封面与编号页
\title{《数据库系统》课程设计报告}
\author{18340206张德龙 \\
	zhangdlong3@mail2.sysu.edu.cn \\
		18340207张昊熹 \\ 
		zhanghx66@mail2.sysu.edu.cn
}
\date{\today}

\usepackage{diagbox}
%\usepackage{fancyhdr}
%\pagestyle{empty} 
{\renewcommand\fcolorbox[4][]{\textcolor{cyan}{\strut#4}} %屏蔽汇编语言mint的错误
\graphicspath{{figure/}}

\let\algorithm\relax  
\let\endalgorithm\relax 
\usepackage[linesnumbered,ruled,lined]{algorithm2e}
\usepackage{algpseudocode}  
\renewcommand{\algorithmicrequire}{\textbf{Inc\_puct:}}   
\renewcommand{\algorithmicensure}{\textbf{Output:}}   
\SetKwFor{For}{for}{do}{endfor}
\newcommand{\ret}{\textbf{return}}   
\usepackage{ulem}

\makeatletter
\newenvironment{breakablealgorithm}
{% \begin{breakablealgorithm}
	\begin{center}
		\refstepcounter{algorithm}% New algorithm
		\hrule height.8pt depth0pt \kern2pt% \@fs@pre for \@fs@ruled
		\renewcommand{\caption}[2][\relax]{% Make a new \caption
			{\raggedright\textbf{\ALG@name~\thealgorithm} ##2\par}%
			\ifx\relax##1\relax % #1 is \relax
			\addcontentsline{loa}{algorithm}{\protect\numberline{\thealgorithm}##2}%
			\else % #1 is not \relax
			\addcontentsline{loa}{algorithm}{\protect\numberline{\thealgorithm}##1}%
			\fi
			\kern2pt\hrule\kern2pt
		}
	}{% \end{breakablealgorithm}
		\kern2pt\hrule\relax% \@fs@post for \@fs@ruled
	\end{center}
}
\makeatother

\newcounter{Emp}[subsubsection]	% 设置计数器
\newcommand{\kuohao}[1]{ \noindent (#1)}

\usepackage{amssymb}% http://ctan.org/pkg/amssymb
\usepackage{pifont}% http://ctan.org/pkg/pifont
\newcommand{\cmark}{\ding{51}}%
\newcommand{\xmark}{\ding{55}}%
\newcommand{\yuan}{\ding{109}}%

\usepackage{threeparttable}
\setcounter{tocdepth}{3}

\begin{document}
	\maketitle 
%{
%	\centering \kaishu 数据科学与计算机学院 \\
%	 计科 18340206张德龙 \quad \quad \quad \quad 计科 18340207张昊熹 \\
%	 zhangdlong3@mail2.sysu.edu.cn  \quad zhanghx66@mail2.sysu.edu.cn\\
%} 
\begin{table}[H]
\centering
\label{tab:my-table}
\begin{threeparttable}
\begin{tabular}{|c|c|c|c|}
\hline
\textbf{题目} & \multicolumn{3}{c|}{人事工资管理系统的设计}        \\ \hline
\textbf{姓名} & \textbf{学号} & \textbf{班级} & \textbf{分工} \\ \hline
张德龙         & 18340206    & 教务3班        & xxx         \\ \hline
张昊熹         & 18340207    & 教务3班        & xxx         \\ \hline
\end{tabular}
\begin{tablenotes}
%    \footnotesize
    \centering
    \item 提交时间: 2021  年  1    月   10    日
   % \item[**] my website is ... %此处加入注释**信息
\end{tablenotes}
\end{threeparttable}
\end{table}

%\begin{abstract}
%	在本次实验中,我们实现了结合神经网络的蒙特卡洛树搜索算法——即Alpha Zero强化学习算法。并且,在实现了基础的Alpha Zero算法之上,我们通过将代码cython化、使用数据增强、将棋盘正则化等方式对模型进行了改进,在训练了120个epoch后,我们实现的模型对阵随机玩家基本能够达到百分百的胜率,对阵仿真次数比自己多一倍的纯蒙特卡洛搜索玩家也能够达到百分之七十六的胜率,证实了我们实现模型的有效性。
%	
%	\keywords{强化学习 \quad 蒙特卡洛树搜索 \quad  AlphaZero \quad cython \quad 数据增强 }
%\end{abstract}
%
%\tableofcontents

\section{开发环境与开发工具}
1、操作系统:Windows 10;

2、DBMS :mysql 8.0.21;

3、编程环境:Java Development Kit 13.0.1;

4、数据库JDBC接口:mysql-connector-java-8.0.22;

5、开发IDE:IDEA 2020.2.4。
\section{系统需求分析}

企业可以通过人事工资管理系统实现对企业人员信息及工资信息的管理,该系统具有一定的人事档案管理功能。经过调研,对企业进行人事工资事务的管理流程、题设要求进行分析,本实验设计的人事工资管理系统具有如下功能:
\begin{enumerate}
	\item 系统的用户管理:包括普通员工、财务管理员、领导管理员的添加、删除,密码修改等;
	\item 员工的信息管理:包括员工基本信息的查询、添加、删除、修改等;
	\item 员工的考勤管理:包括员工考勤情况的查询、添加、删除、修改等;
	\item 部门的信息管理:包括部门的查询、添加、删除、修改等;
	\item 员工的薪资管理:包括薪资的计算、查询、修改等;
	\item 薪资的统计报表:包括企业薪资、部门薪资、职工薪资的查询等;
\end{enumerate}

数据字典是系统中各类数据描述的集合,是
进行详细的数据收集和数据分析所获得的主要成果。教材中定义的数据字典通常包括数据项、数据结构、数据流、数据存储和处理过程5个部分。本次实验中我们小组设计的数据字典如下:

\subsection{主要数据项}
\begin{table}[H]
\centering
\caption{数据项表}
\begin{tabular}{cccc}
\hline
\textbf{名称} & \textbf{别名} & \textbf{描述}       & \textbf{定义} \\ \hline
姓名          & 名字          & 员工/部门等实体的名字       & 16字符变长字符串   \\ \hline
性别          & -           & 员工的性别             & 2字符变长字符串    \\ \hline
编号          & id、编码       & 系统内各对象的唯一标识符      & 8字符变长字符串    \\ \hline
底薪          & 最低工资        & 一个部门下员工一个月获得的最低工资 & 整型          \\ \hline
职务薪资        & 职薪          & 在部门下所任职务的薪资       & 整型          \\ \hline
缺勤惩罚        & -        & 员工没有按时出勤的惩罚       & 整型          \\ \hline
税费          & 税费记录        & 员工需要交纳的税务费用       & 整型          \\ \hline
部门人数        & -           & 一个部门所有员工的数目       & 整型          \\ \hline
时间          & -           & 记录时间              & datetime类型  \\ \hline
考勤状态        & 考勤          & 记录考勤的状态           & 8字符变长字符串    \\ \hline
密码          & password    & 登录管理系统需要的密码       & 32字符变长字符串   \\ \hline
权限          & 权限级别        & 在管理系统中具有的权限       & 32字符变长字符串   \\ \hline
            &             &                   &             \\ \hline
\end{tabular}
\end{table}

\subsection{主要数据结构}
\begin{table}[H]
\centering
\caption{数据结构表}
\label{t2}
\begin{tabular}{ccc}
\hline
\textbf{名称} & \textbf{描述}  & \textbf{定义}         \\ \hline
考勤信息        & 考勤信息的整合      & 考勤状态+时间+编号+员工编号     \\ \hline
员工信息        & 存储员工的信息      & 姓名+性别+编号            \\ \hline
部门信息        & 存储部门相关信息     & 人数+底薪+名称+编号         \\ \hline
管理员信息       & 存储管理系统的管理员信息 & 姓名+密码+权限+编号         \\ \hline
职务信息        & 存储员工职务相关信息   & 职务薪资+职务名称+员工编号+部门编号 \\ \hline
历史工资记录    &存储过去每个月的工资记录 & 时间+员工/部门编号+税费+税后工资                    \\ \hline
\end{tabular}
\end{table}
\subsection{主要数据流}
\kuohao{1} 员工信息变更数据流:
\begin{itemize}
\item 说明:员工信息的变更产生的数据流向;
\item 数据流来源:员工信息变动事务;
\item 数据流去向:人事管理事务、薪资结算事务;
\item 平均流量:每月一次;
\item 高峰期流量:每月十几次。
\end{itemize}

\kuohao{2} 考勤信息变更数据流:
\begin{itemize}
\item 说明:考勤信息的变更产生的数据流向;
\item 数据流来源:日常考勤事务;
\item 数据流去向:人事管理事务、薪资结算事务;
\item 平均流量:每天几百次;
\item 高峰期流量:每天几千次。
\end{itemize}


\kuohao{3} 部门信息变更数据流:
\begin{itemize}
\item 说明:考勤信息的变更产生的数据流向;
\item 数据流来源:部门信息变更事务;
\item 数据流去向:人事管理事务、薪资结算事务;
\item 平均流量:每月一次;
\item 高峰期流量:每月十几次。
\end{itemize}

\subsection{数据存储}
\kuohao{1} 考勤信息表:
\begin{itemize}
\item 说明:考勤信息的整合;
\item 数据结构:考勤信息;
\item 数据量:一天约100 $\times$ 32 = 3200字节。
\end{itemize}

\kuohao{2} 员工信息表:
\begin{itemize}
\item 说明:员工信息的整合;
\item 数据结构:员工信息;
\item 数据量:约100 $\times$ 32 = 320字节 。
\end{itemize}

\kuohao{3} 部门信息表:
\begin{itemize}
\item 说明:部门信息的整合;
\item 数据结构:部门信息;
\item 数据量:约10 $\times$ 32 = 320字节 。
\end{itemize}

\kuohao{4} 管理员信息表:
\begin{itemize}
\item 说明:管理员信息的整合;
\item 数据结构:管理员信息;
\item 数据量:约10 $\times$ 32 = 320字节 。
\end{itemize}

\kuohao{5} 职务信息表:
\begin{itemize}
\item 说明:职务信息的整合;
\item 数据结构:职务信息;
\item 数据量:约100 $\times$ 64 = 6400字节 。
\end{itemize}

\kuohao{6} 历史薪资表:
\begin{itemize}
\item 说明:公司往月发放的薪资记录;
\item 数据结构:工资信息;
\item 数据量:每月约100 $\times$ 64 = 6400字节 。
\end{itemize}

\subsection{主要处理过程}

处理过程名:实时计算薪资。

\begin{itemize}
\item 说明:因为考勤、员工信息的不断变更,需要能根据信息实时计算薪资;
\item 输入:员工信息表、部门信息表、职务信息表、时间、考勤状态表;
\item 输出:员工薪资表;
\item 处理:结合员工的考勤、职务。部门、输入的时间自动求出工资。
\end{itemize}



\section{功能需求分析}

企业人事工资管理系统按照上面所述,管理功能的需求比较清晰,主要实现了对员工、部门、员工的考勤、员工的薪资等的管理。系统功能模块如图\ref{gongneng}所示。 \par 
	其中“信息管理”板块中的每一个功能管理项都包括查看、添加、删除和修改功能。
\begin{figure}[H]
    \centering
    \includegraphics[width=1\linewidth]{gongneng1}
    \caption{系统功能模块图}
    \label{gongneng}
\end{figure}
\section{系统设计}
\subsection{数据概念结构设计}
\begin{enumerate}
	\item 数据流程图。如图\ref{shuju}所示。
	\item 系统ER图。经调研分析后,得人事工资管理系统整体基本ER图如图\ref{ER}所示;
	\item Mysql中的加强ER图(EER图),通过ER图使用mysql建立的加强版本的ER图如下图\ref{EER}所示。
\end{enumerate}
\begin{figure}[H]
    \centering
    \includegraphics[width=1\linewidth]{shuju}
    \caption{系统功能模块图}
    \label{shuju}
\end{figure}
\begin{figure}[H]
    \centering
    \includegraphics[width=1\linewidth]{ERGraph}
    \caption{ER图}
    \label{ER}
\end{figure}

\subsection{数据库关系模式设计}
按照ER图到逻辑关系模式的转换规则,可得到系统如下6个关系。
\begin{enumerate}
	\item 员工信息(\uline{员工编号},姓名,性别);
	\item 部门信息(\uline{部门编号},名称,底薪,人数);
	\item 管理信息(\uline{职务薪资},职务名称,\uline{部门编号},\uline{员工编号});
	\item 考勤信息(\uline{时间},考勤状态,\uline{员工编号});
	\item 历史薪资信息(\uline{薪资编号},日期,员工编号,员工姓名,员工部门,职务,底薪,职务薪资,缺勤惩罚,税前薪资,税费,税后薪资);
	\item 系统管理员信息(\uline{管理员编号},管理员名称,密码,权限);
\end{enumerate}


\subsection{数据库物理结构设计}
本系统数据库表的物理设计通过使用mysql workbench的EER图导出的创建表SQL命令来实现,根据ER图建立的EER关系模型如下:
\begin{figure}[H]
    \centering
    \includegraphics[width=1\linewidth]{EERGraph}
    \caption{MySQL中的EER图}
    \label{EER}
\end{figure}
生成的创建数据库表的SQL命令主要部分如下:
\begin{lstlisting}[language=SQL]
CREATE TABLE IF NOT EXISTS `EnterpriseDB`.`Management` (
  `salary` INT NULL,
  `job` VARCHAR(16) NOT NULL,
  `department_id` VARCHAR(8) NOT NULL,
  `worker_id` VARCHAR(8) NOT NULL,
  PRIMARY KEY (`job`, `department_id`, `worker_id`))
ENGINE = InnoDB;
\end{lstlisting}


\subsection{主要创新点}

为了加强性能、更好地完成客户需求,在系统的设计中我们使用学习到的数据库知识在实验基础的要求上进行了创新:

%\subsubsection{防sql注入攻击}

\subsubsection{设置了多个触发器}
为了实现数据库的完整性约束、自动更新等需求,在本次实验中我们对相关表设置了多个触发器实现了如下需求:

\kuohao{1} 对变量进行完整性约束:

例如,考勤状态必须为正常/迟到/缺勤/请假之一,需要设置如下触发器保证完整性:
\begin{lstlisting}[language=SQL]
CREATE DEFINER = CURRENT_USER TRIGGER `EnterpriseDB`.`Attendence_BEFORE_INSERT` BEFORE INSERT ON `Attendence` FOR EACH ROW
BEGIN
	if new.state != '正常' and new.state != '迟到' 
    and new.state != '缺勤' and new.state != '请假'
	then signal sqlstate '45000'
    set message_text = '状态必须为正常/迟到/缺勤/请假之一';
	end if;
END
CREATE DEFINER = CURRENT_USER TRIGGER `EnterpriseDB`.`Attendence_BEFORE_UPDATE` BEFORE UPDATE ON `Attendence` FOR EACH ROW
BEGIN
	if new.state != '正常' and new.state != '迟到' 
    and new.state != '缺勤' and new.state != '请假'
	then signal sqlstate '45000'
    set message_text = '状态必须为正常/迟到/缺勤/请假之一';
	end if;
END
\end{lstlisting}
其余的完整性约束如性别必须为男或女也在为相应表设置了触发器。

\vspace{1em}
\kuohao{2} 对于数据的自动更新:

在插入/删除某个员工信息后,我们希望能够在相应部门的部门人数数据项进行更改,因此我们设置了如下触发器进行实现:
\begin{lstlisting}[language=SQL]
CREATE DEFINER = CURRENT_USER TRIGGER `EnterpriseDB`.`Management_BEFORE_INSERT` BEFORE INSERT ON `Management` FOR EACH ROW
BEGIN
	if new.department_id not in (select id from Department)
	then signal sqlstate '45000'
    set message_text = '员工部门信息输入错误——不存在的部门';
	end if;
    update Department set member_num = member_num + 1 where id = new.department_id;
END
CREATE DEFINER = CURRENT_USER TRIGGER `EnterpriseDB`.`Management_BEFORE_DELETE` BEFORE DELETE ON `Management` FOR EACH ROW
BEGIN
	update Department set member_num = member_num - 1 where id = old.department_id;
END
\end{lstlisting}
注意到员工调整部门(更新操作)需要对两个部门的人数进行更改:
\begin{lstlisting}[language=SQL]
CREATE DEFINER = CURRENT_USER TRIGGER `EnterpriseDB`.`Management_BEFORE_UPDATE` BEFORE UPDATE ON `Management` FOR EACH ROW
BEGIN
	if new.department_id not in (select id from Department)
	then signal sqlstate '45000'
    set message_text = '员工部门信息输入错误——不存在的部门';
	end if;
    update Department set member_num = member_num + 1 where id = new.department_id;
	update Department set member_num = member_num - 1 where id = old.department_id;
END
\end{lstlisting}

\subsubsection{建立视图}
工资的相关信息分布在不同的表中,而且每月的工资会随考勤情况,不停的变动,所以通过维护视图,能够在展示数据的情形下,保持数据动态性,且避免数据冗余。\par
使用视图的优点是将细节封装。从下述代码可以看出,为了实时展现薪资,视图要满足以下几点;
\begin{enumerate}
	\item 将全部考勤信息中的当月信息筛选出来,这通过“MONTH(time)=MONTH(NOW()) and YEAR(time)=YEAR(NOW()”实现。
	\item 统计缺勤次数,这看似简单,可以通过聚集函数COUNT()结合GROUP BY子句实现。但实际要求中,为缺勤0次的用户也显示id,所以要用到外连接进行NULL的补充和IFNULL()函数的转换。
	\item 计算薪资涉及4张表的笛卡尔积,并含有列之间的加减乘除运算。
\end{enumerate}
\begin{lstlisting}[language=SQL]
CREATE VIEW `Salary` AS 
with Checkin(id, LateCount) as 
	(select worker_id, IFNULL(num, 0) 
    from attendence LEFT OUTER JOIN 
   (select worker_id, count(state) from attendence 
		where state != '正常' and MONTH(time)=MONTH(NOW()) and YEAR(time)=YEAR(NOW())
		group by worker_id) as LateCount(id, num) 
    on attendence.worker_id = LateCount.id group by worker_id)
select Worker.id, Worker.name, Department.name as department_name, salary, base_salary, -100*Checkin.LateCount as LatePunish, salary+base_salary-100*Checkin.LateCount as total_salary 
from Worker join Management join Department join Checkin 
where Worker.id = Management.worker_id and Management.department_id = Department.id and Worker.id = Checkin.id;
\end{lstlisting}
对于薪资存储,可见查询的开销很大,是否应该在物理结构中重新设计一张表,来记录这些信息?我们进行了权衡。首先,计算薪资的所需信息都存储在各个不同的表中,若是重新存储一张表会造成大量的信息冗余。其次,考虑到工资表占用的存储空间较大,并且访问相对不频繁,只在每个月发放薪资是需要查询,所以在权衡中,我们认为视图这一展现方式更合理。
\par 物化视图能进一步提升性能,但对于需要动态变化的表并不适用。
\subsubsection{建立存储过程}
建立存储过程,完成向历史工资记录批量写入的任务。每个月调用一次该存储过程,将会把动态变化的每月工资记录,记录到新表中。建立存储过程便于调用。\par 
存储过程的编写思路很直观,遍历视图salary中每一行,根据税前工资计算税费,然后再将相关信息存储历史薪资记录表SalaryHistory。\par 
值得一提的是,对于历史薪资记录表SalaryHistory的主码SHid,我们设置了主码自增,这样每次插入,数据库都会自动对编号+1,方便了插入。在Workbench中选中列的“AI”属性就能实现。
\begin{lstlisting}[language=SQL]
delimiter //
create procedure saveHistorySalary()
BEGIN
	DECLARE n INT DEFAULT 0;
    DECLARE date DATETIME;
	DECLARE i INT DEFAULT 0;
	DECLARE tax INT DEFAULT 0;
	DECLARE Tsalary INT DEFAULT 0;
	SELECT COUNT(*) FROM salary INTO n;
	SET i=0;
    SET date = now();
	WHILE i<n DO 
		set TAX = 0;
        SELECT total_salary FROM salary LIMIT i,1 into Tsalary ;
        if Tsalary <= 10000
    		then set tax = 0;
    	elseif Tsalary <= 20000
    		then set tax = (Tsalary-10000)*0.05;
    	else set tax = 500 + (Tsalary-20000)*0.1;
    	end if;
    	INSERT INTO SalaryHistory(date, worker_id, worker_name, department_name, job, base_salary, salary, punishment, before_tax, tax, total_salary) 
        SELECT date, id, name, department_name, job, base_salary, salary, LatePUnish, total_salary, tax, total_salary-tax FROM salary LIMIT i,1;
  		SET i = i + 1;
	END WHILE;
END //
delimiter ;
\end{lstlisting}
\subsubsection{创建索引}
根据数据字典中对于部门的分析可得,Department表的主码是部门id,难以记忆。但对于该部门管理表的查询又会比较频繁,于是对于部门名的查询就会较为频繁,因此为Department表上的name使用如下语句创建索引。
\begin{lstlisting}[language=SQL]
create index NAMEINDEX on Department(name);
\end{lstlisting}
从而使得对于部门名的查询更为高效。

\section{系统功能的实现}

\subsection{数据库连接通用模块}

要使得应用程序能够与数据库通信,首先需要构建数据库连接的通用模块。

\subsubsection{实现思想}
JDBC标准定义了Java程序连接数据库服务器的应用程序接口,通过查询Mysql官网提供的手册\url{https://dev.mysql.com/doc/connector-j/8.0}来实现数据库的连接。

\vspace{1em}

\kuohao{1} 连接到数据库:

首先建立一个通过调用DriverManager类的getConnection方法来打开一个数据库连接:
\begin{lstlisting}[language=java]
Class.forName("com.mysql.cj.jdbc.Driver");
conn = DriverManager.getConnection("jdbc:mysql://localhost:3306" + "/" + ConfigIni.DBName +
        "?" + "user=" + ConfigIni.user + "&password=" + ConfigIni.passwd + "&serverTimezone=UTC");
\end{lstlisting}

\kuohao{2} 向数据库中传递SQL语句:

对执行查询、更新等SQL语句的方法进行了封装,便于获取需要的结果。更新类SQL语句需要获取执行是否成功的结果,JDBC设置为int类型;查询类SQL语句执行返回值为ResulSet结果集类型,并且在执行更新语句完毕后可以将statement释放。执行更新类SQL语句的关键代码如下,对查询类SQL命令的封装类似:
\begin{lstlisting}[language=java]
// 执行更新类SQL命令的函数
public static int executeUpdate (String sql) {
    int i= 0;
    try {
        stmt = conn.createStatement (ResultSet. TYPE_SCROLL_SENSITIVE,
        ResultSet. CONCUR_READ_ONLY);
        i = stmt.executeUpdate (sql);
        //conn.commit ();
    }
    catch(Exception e) {
        e.printStackTrace();
    }
    finally {
        free_Stmt(stmt);
    }
    return i;
}
\end{lstlisting}

\vspace{1em}
\kuohao{3} \textbf{防SQL注入}——使用预备语句:

在查询手册的时候调查到,使用预备语句能够有效地防止SQL注入攻击。因此,在数据库连接中我们封装了使用预备语句的接口便于调用,其中封装的执行查询预备语句的函数如下:
\begin{lstlisting}[language=java]
// 执行查询预备SQL命令,返回记录集对象函数
    public static ResultSet executeQuery (PreparedStatement stmt) throws SQLException{
        try {
            rs= stmt.executeQuery();
        }catch(SQLException ex) {
            handleSQLException(ex);
            throw ex;
        }
        // 注意:free statement 会也会close result
        return rs;
    }
\end{lstlisting}

\noindent
预备语句的使用举例如下:
\begin{lstlisting}[language=java]
String sql = "SELECT * FROM Administrator WHERE name=? AND passwd=?";
// 防注入攻击
// 预编译
PreparedStatement stmt = Connector.conn.prepareStatement(sql);
//设置参数
stmt.setString(1, user);
stmt.setString(2, passwd);
// 执行
ResultSet rs = Connector.executeQuery(stmt);
\end{lstlisting}
其中setString方法会将用户输入的符号进行转义,有效地防止了SQL注入攻击。

\vspace{1em}
\kuohao{4} 其余辅助函数:

我们编写的数据库连接类还封装了许多辅助函数便于后续功能的调用,其中关键辅助函数的签名如下:
\begin{lstlisting}[language=java]
// 打印SQL语句执行中的问题
private static void handleSQLException(SQLException ex)
// 关闭Connector
public static void close()
// 释放结果集
private static void free_Res(ResultSet rs)
// 释放语句
public static void free_Stmt(Statement stmt)
// 获取结果集的行数
public static int getRowCnt(ResultSet rs)
\end{lstlisting}

\subsection{登陆界面模块}

\subsubsection{实现思想}
核心思想使用上述封装的Connector连接数据库,然后通过调用预备语句
\begin{lstlisting}[language=SQL]
SELECT * FROM Administrator WHERE name=? AND passwd=?
\end{lstlisting}
来查询是否输入了在数据库中存储的用户名和密码,关键代码如下:
\begin{lstlisting}[language=java]
String sql = "SELECT * FROM Administrator WHERE name=? AND passwd=?";
// 防注入攻击
// 预编译
PreparedStatement stmt = Connector.conn.prepareStatement(sql);
// 设置参数
stmt.setString(1, user);
stmt.setString(2, passwd);
// 执行
ResultSet rs = Connector.executeQuery(stmt);
if(rs.next())
{
    System.out.println("密码正确");
    jf.setVisible(false);
    Main mainPage = new Main(); //开始主题运行界面
    mainPage.main(new String [] {});
}		
\end{lstlisting}


\subsubsection{运行界面}
密码输入界面如下:
\begin{figure}[H]
	\centering
	\begin{minipage}[t]{0.3\linewidth}
		\centering
		\includegraphics[width=1\linewidth]{login2}
		\caption{密码错误提示}
	\end{minipage}
	\begin{minipage}[t]{0.29\linewidth}
		\centering
		\includegraphics[width=1\linewidth]{login1}
		\caption{密码输入界面}
	\end{minipage}
	\begin{minipage}[t]{0.3\linewidth}
		\centering
		\includegraphics[width=1\linewidth]{login3}
		\caption{密码不能为空}
	\end{minipage}
\end{figure}
密码输入正确后,进入主界面模块。

\subsection{主界面模块}
\subsubsection{实现思想}
主界面的GUI处理一些交互逻辑,生成相应的sql语句。将这些sql语句传入封装后的Connector,得到返回的查询表或其他结果。再将查询表中的信息提取,展现在GUI界面。\par 
本实验的GUI界面通过Java的Swing框架实现,最核心的步骤是取得查询表的元数据:
\begin{lstlisting}[language=java]
PreparedStatement stmt = Connector.conn.prepareStatement(sql);
ResultSet rs = Connector.executeQuery(stmt);
ResultSetMetaData rsmd = rs.getMetaData();
\end{lstlisting}
再依次插入到JTable组件中:
\begin{lstlisting}[language=java]
Object[] title = {};
tableModel.setRowCount(0);    //清空表格中的数据
for(int i = 1; i <= rsmd.getColumnCount(); i++) {
    title = appendValue(title, rsmd.getColumnName(i));
}
while(rs.next()) {
    Object[] row = {};
    for(int i = 1; i <= rsmd.getColumnCount(); i++) {
        row = appendValue(row, rs.getString(i));
    }
    tableModel.addRow(row);
}
tableModel.setColumnIdentifiers(title);
table.setRowHeight(30);
table.setModel(tableModel);    //应用表格模型
\end{lstlisting}
\subsubsection{运行界面}
主界面如下,分为三个部分,系统管理、员工信息查询、工资查询。
\begin{figure}[H]
    \centering
    \includegraphics[width=1\linewidth]{main}
    \caption{系统主界面}
\end{figure}
其中员工信息查询有4个功能,分别为员工基本信息查询、部门基本信息查询、员工和任职部门查询、考勤信息查询。这四个功能的界面类似,以部门信息查询为例:
\begin{figure}[H]
    \centering
    \includegraphics[width=1\linewidth]{department}
    \caption{部门信息查询}
\end{figure}
界面上有4个功能,分别为查询、插入、删除、修改。在相应字段输入相应内容即可实现查询和插入、删除功能。对于修改功能,需要先点击希望删除的表项,程序将会记忆这条信息,在左边的文本框显示相应文字,证明已经选中了相应信息,然后再修改希望改变的字段,就能实现修改操作。
\begin{figure}[H]
    \centering
    \includegraphics[width=1\linewidth]{change}
    \caption{修改选中表项}
\end{figure}
对于工资查询部分仅提供查询功能,这是由于每月工资是视图实时计算的,不需要修改,而历史工资记录属于历史也不需要修改,每月工资查询如图:
\begin{figure}[H]
    \centering
    \includegraphics[width=1\linewidth]{salary}
    \caption{每月工资查询}
\end{figure}
可见工资由三个部分构成:salary是职务薪资,base\_salary是部门底薪,LatePunish代表缺勤惩罚,total\_salary是税前工资。只要在相应的字段输入信息,就能实现多种报表查询,如查询当月部门的薪资信息,或某个职位相关人员的信息。 \par
在该页面还有“保存至历史薪资记录表”这一按钮,这一按钮调用了存储过程saveHistorySalary(),将会使用当前的时间,把当前的工资写进历史薪资记录表,并展示历史薪资记录表,保存后效果如下图:
\begin{figure}[H]
    \centering
    \includegraphics[width=1\linewidth]{saveSalary}
    \caption{每月工资查询}
\end{figure}
在历史记录薪资中,除了展示每月薪资部分的信息,还会根据税率分级的计算税费,并输出税后工资。对于10000以下不收税,对于10000-20000的部分收5\%的税,对于20000以上的部门收10\%的税,这个计算过程在存储过程中实现。对于历史薪资查询,需要进入到“查询历史薪资”功能,查询功能与员工信息查询类似,不再展示。\par
更多界面演示请参考附件视频。

\section{实验总结}
(不知道写哪里就写这了),本次实验中我们使用Java语言结合MySQL数据库,通过JDBC接口,编写了一个基础的人事工资管理系统。我们精心设计了关系结构,做到了满足xx范式?尽可能的减少信息冗余。实现过程中,视图的编写最耗费时间,我们多次对视图进行了重构,通过视图的封装实现了计算工资这一复杂的需求。编写GUI过程中,学习JDBC接口与swing框架组件的交互也很需要精力。\par
实验中也遇到了有趣的问题,如workbench导出的数据可能属于不同的编码类型,这会导致字符串比较上的一些错误,需要手动的修改相关类型。
\par 
该系统还有很大的改进空间,例如对于查询功能的优化,支持范围查询,支持多项更新等。
\par 
在本次实验中,我们实现了。

在训练过程中,

除了使用cython对数据采样本身进行加速外,

在对弈过程中,


在本次实验中,

通过本次实验,我们对于马尔可夫决策过程、贝尔曼方程等相关知识的理解更为深刻;对于蒙特卡洛树搜索、代码的cython实现、棋盘的正则化等相关过程更为熟悉;对于前沿的强化学习知识——基于蒙特卡洛树搜索与神经网络的Alpha Zero算法有了更深的体会……总而言之,收获颇丰。



%\begin{table}[H]
%\centering
%\caption{实验结果}
%\begin{threeparttable}
%\begin{tabular}{cccc}
%\hline
%      & Y-搜索总结点数 & N-搜索总结点数 &  搜索结点下降百分比 \\ \hline
%随机排序   &   88556      &  122451   & 27.68 \\ 
%行棋排序 &    5842     &  14312  & \textbf{59.18} \\ 
%行棋排序+向前剪枝 &   \textbf{1631}     &     3617   &  54.91 \\ \hline
%\end{tabular}
%\begin{tablenotes}
%    \footnotesize
%    \centering
%    \item[注:]  Y-表示使用置换表,N-表示不使用置换表。
%   % \item[**] my website is ... %此处加入注释**信息
%\end{tablenotes}
%\end{threeparttable}
%\end{table}

%\begin{figure}[H]
%	\centering
%	\begin{minipage}[t]{0.29\linewidth}
%		\centering
%		\includegraphics[width=1\linewidth]{zeve1}
%		\caption{随机排序}
%	\end{minipage}
%	\begin{minipage}[t]{0.3\linewidth}
%		\centering
%		\includegraphics[width=1\linewidth]{zeve2}
%		\caption{行棋排序}
%	\end{minipage}
%	\begin{minipage}[t]{0.3\linewidth}
%		\centering
%		\includegraphics[width=1\linewidth]{zeve3}
%		\caption{行棋排序+剪枝}
%	\end{minipage}
%\end{figure}

%\maketitle
%\begin{abstract}
%	摘要
%	\keyword key
%\end{abstract}




%\begin{figure}[H]
%	\centering
%	\includegraphics[width=0.8\linewidth]{1}
%	\caption{caption}
%\end{figure}

%\begin{thebibliography}{99}
%	\bibitem{1}  基于模拟的搜索与蒙特卡罗树搜索(MCTS)——\url{https://www.cnblogs.com/pinard/p/10470571.html}
%	\bibitem{3} AlphaGo Zero强化学习原理——\url{https://www.cnblogs.com/pinard/p/10609228.html}
%	\bibitem{4} 马尔科夫决策过程(MDP)——\url{https://www.cnblogs.com/pinard/p/9426283.html}
%	\bibitem{2} Alpha Zero初探——\url{https://www.hhyz.me/2018/08/08/AlphaGO-Zero}
%	\bibitem{Alpha Z} Silver D , Schrittwieser J , Simonyan K , et al. Mastering the game of Go without human knowledge[J]. Nature, 2017, 550(7676):354-359.
%%	\bibitem{3} Alpha Zero 教程——\url{https://github.com/junxiaosong/AlphaZero_Gomoku}
%%	\bibitem{4} Alpha Zero 教程——\url{https://github.com/bhansconnect/alpha_zero_othello}
%%	\bibitem{5} Apl
%\end{thebibliography}
%
%\begin{appendices}
%	\section{组员分工}
%	\begin{table}[H]
%		\centering
%		\caption{分工明细}
%		\label{tab:F1}
%		\begin{tabular}{l|l|l|l}
%			\hline
%			& \multicolumn{3}{c}{\textbf{分工}}                                                                                                                   \\ \hline
%			\textbf{张德龙} & \multicolumn{3}{l}{\begin{tabular}[c]{@{}l@{}}负责模型原理设计、模型代码的主要实现、辅助函数实现、报告编写。
%		
%			\end{tabular}} \\ \hline
%			\textbf{张昊熹} & \multicolumn{3}{l}{\begin{tabular}[c]{@{}l@{}}负责模型原理设计、模型代码实现、辅助函数主要实现、报告编写。
%			\end{tabular}}                                  \\ \hline
%
%		\end{tabular}
%	\end{table}
%\end{appendices}

\end{document}